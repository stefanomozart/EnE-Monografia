Este capítulo oferece sugestões para produção de um documento descrevendo um Trabalho de Conclusão 
de curso.

\section{UnB}%
A \acrlong{UnB} oferece diversas informações em seu 
sítio\footnote{\url{http://www.unb.br/oportunidades/projeto_final_de_curso}}. O texto existente em 
21/11/2014 é reproduzido a seguir:

\fontshape{it}\selectfont%
Os cursos de graduação, especialização e pós-graduação têm como objetivo formar o aluno e 
prepará-lo para o exercício profissional. Como avaliação do aprendizado, a universidade exige um 
projeto que mobiliza os estudantes a colaborar com a pesquisa acadêmica. Desde a escolha do tema 
até a apresentação do trabalho final, o tempo do aluno é ocupado quase integralmente. Para 
facilitar a vida desses estudantes, o Portal UnB preparou uma série de dicas de professores 
especialistas no assunto.

\subsection{Os tipos}
A monografia, a dissertação e a tese são, respectivamente, os trabalhos de conclusão de curso de 
graduação ou especialização, mestrado e doutorado. A grande diferença é a profundidade exigida no 
projeto, aumentada de acordo com a importância do título de cada nível acadêmico. Mas, em todos os 
casos, a pesquisa deve abordar o tema selecionado com coerência, consistência e referencial teórico 
adequado.

Alguns cursos de graduação não exigem monografia, mas um relatório de estágios realizados, como 
acontece nas licenciaturas. A metodologia de pesquisar e apresentar resultados se mantém, como é 
exigido em todo projeto final.

Uma monografia é, genericamente, um relatório de pesquisa sobre o assunto estudado. É específico a 
um tema pré-definido dentro de uma área de conhecimento e aborda questões e análises de um 
problema, a construção de uma teoria ou o desenvolvimento de um produto.

Exigida no mestrado, a dissertação cobra do futuro mestre um conhecimento mais profundo. A pesquisa 
deve ser o resultado em relatório que representa o trabalho experimental ou exposição científica 
com um tema bem delimitado, e demonstrar o conhecimento de literatura existente sobre o assunto.

A mais densa entre todos os projetos finais, a tese de doutorado exige mais no que diz respeito a 
teoria e metodologia do tema pesquisado. Deve apresentar contribuições reais para o desenvolvimento 
específico da especialidade em questão. A base do estudo demanda uma investigação original.

\subsection{Teoria e prática}
Todo projeto de conclusão de curso exige um relatório escrito baseado em teorias, mesmo que o 
assunto estudado seja algo prático como uma campanha publicitária ou um projeto arquitetônico. 
Porém, o inverso não se aplica.

As divisões dos tipos de trabalho variam entre cada área de conhecimento. Em suma, o projeto pode 
ser teórico, prático ou uma união dos dois. Na primeira situação, o aluno pode fazer estudo de caso 
- pesquisar sobre um fato histórico ou evento importante - ou formular uma teoria - por meio de 
pesquisa ou reavaliação das semelhantes.

O projeto prático se dedica a criação e construção de um produto, que pode variar de um novo motor a uma composição musical. O curso de graduação costuma oferecer a opção de um trabalho prático aos alunos. No caso dos cursos de mestrado e doutorado, nem todos os departamentos da universidade dispõem de linhas de pesquisa que permitam um projeto que vá além da teoria acadêmica.

A união dos dois gêneros é comum quando o universitário relata a experiência de estágio ou na simulação de um projeto, como a construção de maquetes ou esquemas computacionais. As opções são vastas e o aluno deve explicar como e o que se deve fazer para que o projeto se torne possível.

\subsection{Começo do projeto}
Parece óbvio, mas muitos alunos esquecem a questão principal na hora de escolher o tema: o assunto deve interessar e estimular a pesquisa. Conviver meses com um tema que não agrada torna o trabalho mais complicado. Porém, escolher um bom tema não é abraçar e desenvolver sobre tudo que ele é e engloba. É preciso delimitar o assunto de forma específica.

Um trabalho sobre a história do mundo, por exemplo, está fadado a se tornar superficial. Além de extremamente amplo, é grande o volume de informações a ser levantado e estudado. É importante ter foco para desenvolver um projeto coeso e com credibilidade.

Além disso, o estudante necessita desenvolver um problema e traçar uma hipótese. Em um exemplo bem simples: a Guerra no Iraque (tema) e o terrorismo mundial (problema) – o aumento dos ataques depois da invasão americana (hipótese); ou seja, o que o aluno quer tratar e onde ele espera chegar na pesquisa. A não comprovação da hipótese não inviabiliza o trabalho, desde que o desenvolvimento da análise enriqueça os conhecimentos sobre o tema tratado.

A prática essencial para o desenvolvimento de qualquer projeto é a pesquisa bibliográfica. As consultas às bibliotecas respaldam a parte teórica do estudo e podem elucidar diversas questões, sejam específicas do projeto ou sobre metodologias científicas. Nesse ponto, o papel do professor orientador é fundamental para a condução da pesquisa. Além da seleção dos livros, o docente analisa as melhores possibilidades de desenvolver o assunto, em todas as suas fases. Ele também pode indicar a aplicação de entrevistas e outros elementos de apoio ao conteúdo do projeto.

Atualmente, o meio mais difundido de pesquisa é a Internet. Além de facilitar o acesso a documentos, pela rede é possível saber quanto o tema escolhido já foi objeto de estudo de outros acadêmicos. Mas essa facilidade deve ser utilizada para indicar um caminho.

\subsection{Estrutura e regras}
Antes do próprio trabalho escrito, o estudante deve fazer um projeto ou plano de pesquisa. O documento identifica o que deve ser feito, o porquê, como e onde será realizado o levantamento. Não há um modelo rígido para a apresentação do projeto de pesquisa, mas os seguintes elementos devem ser respondidos no texto:

\begin{enumerate}
  \item Definição do objeto de estudo (tema/problema da pesquisa)
  \item Justificativa
  \item Hipóteses de trabalho
  \item Discussão teórica
  \item Metodologia
  \item Pesquisa Bibliográfica
\end{enumerate}

Seja monografia, dissertação ou tese, a parte escrita possui uma estrutura semelhante, embora cada uma tenha características próprias referentes à profundidade do tema estudado.

De acordo com a Associação Brasileira de Normas Técnicas (ABNT), um trabalho acadêmico deve englobar os elementos pré-textuais (como resumo e índice), pós-textuais (bibliografia, anexos, entre outros) e textuais. Esses últimos compõem a parte central do trabalho - introdução, desenvolvimento e conclusão.

A introdução é a parte inicial do texto e deve constar o objeto de pesquisa, os objetivos, a justificativa da escolha do tema e outras informações que sejam necessárias para esclarecer o assunto.

A parte principal do trabalho está concentrada no desenvolvimento. É uma exposição sistematizada e ordenada de toda o estudo desenvolvido, apresentando análise e interpretação das informações e dados obtidos. A conclusão é a etapa final do texto. Nela, são apresentados os resultados tendo como referência os objetivos e hipóteses da pesquisa.

Em todo o trabalho a linguagem utilizada deve ser interessante, sem apelar para a linguagem coloquial. O trabalho deve estar de acordo com as normas da ABNT. Procure livros sobre estrutura e regras do tipo de projeto final específico de seu interesse.

\subsection{O projeto está pronto. E agora?}
Após a finalização do projeto, chega o momento de preparar a apresentação. Em geral, a banca examinadora é formada por três docentes, sendo um deles o professor orientador do projeto. Também é comum aos alunos o direito de escolha dos avaliadores, desde que seja pertinente ao assunto e ao objetivo do estudo.

Esses professores recomendam uma apresentação resumida do projeto, pontuando as características essenciais e como se chegou às conclusões. É sempre bom explicar o cronograma de todo o trabalho. É preciso, também, ficar atento ao tempo. Não é necessário explicar os conceitos já citados no projeto e pode influenciar a nota final. Lembre-se que as explicações são voltadas para os avaliadores, que já leram o seu trabalho.

Durante as considerações da banca examinadora não se deve interromper a avaliação dos professores, exceto quando eles dirigirem diretamente uma pergunta ao aluno. Educação e conhecimento dos procedimentos acadêmicos são essenciais para uma boa apresentação. Após a avaliação, os professores pedem para os presentes se retirarem da sala. É feita uma reunião onde será decidida a nota do projeto.

Cada departamento possui regras e orientações para a apresentação dos trabalhos de conclusão. Cabe ao aluno perguntar à coordenação do curso e ao orientador todas as etapas do processo de elaboração do projeto final.

\fontshape{n}\selectfont%